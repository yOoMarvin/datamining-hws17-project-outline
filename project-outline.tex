% do not change these two lines (this is a hard requirement
% there is one exception: you might replace oneside by twoside in case you deliver 
% the printed version in the accordant format
\documentclass[11pt,titlepage,oneside,openany]{article}
\usepackage{times}


\usepackage{graphicx}
\usepackage{latexsym}
\usepackage{amsmath}
\usepackage{amssymb}

\usepackage{ntheorem}

% \usepackage{paralist}
\usepackage{tabularx}

% this packaes are useful for nice algorithms
\usepackage{algorithm}
\usepackage{algorithmic}

% well, when your work is concerned with definitions, proposition and so on, we suggest this
% feel free to add Corrolary, Theorem or whatever you need
\newtheorem{definition}{Definition}
\newtheorem{proposition}{Proposition}


% its always useful to have some shortcuts (some are specific for algorithms
% if you do not like your formating you can change it here (instead of scanning through the whole text)
\renewcommand{\algorithmiccomment}[1]{\ensuremath{\rhd} \textit{#1}}
\def\MYCALL#1#2{{\small\textsc{#1}}(\textup{#2})}
\def\MYSET#1{\scshape{#1}}
\def\MYAND{\textbf{ and }}
\def\MYOR{\textbf{ or }}
\def\MYNOT{\textbf{ not }}
\def\MYTHROW{\textbf{ throw }}
\def\MYBREAK{\textbf{break }}
\def\MYEXCEPT#1{\scshape{#1}}
\def\MYTO{\textbf{ to }}
\def\MYNIL{\textsc{Nil}}
\def\MYUNKNOWN{ unknown }
% simple stuff (not all of this is used in this examples thesis
\def\INT{{\mathcal I}} % interpretation
\def\ONT{{\mathcal O}} % ontology
\def\SEM{{\mathcal S}} % alignment semantic
\def\ALI{{\mathcal A}} % alignment
\def\USE{{\mathcal U}} % set of unsatisfiable entities
\def\CON{{\mathcal C}} % conflict set
\def\DIA{\Delta} % diagnosis
% mups and mips
\def\MUP{{\mathcal M}} % ontology
\def\MIP{{\mathcal M}} % ontology
% distributed and local entities
\newcommand{\cc}[2]{\mathit{#1}\hspace{-1pt} \# \hspace{-1pt} \mathit{#2}}
\newcommand{\cx}[1]{\mathit{#1}}
% complex stuff
\def\MER#1#2#3#4{#1 \cup_{#3}^{#2} #4} % merged ontology
\def\MUPALL#1#2#3#4#5{\textit{MUPS}_{#1}\left(#2, #3, #4, #5\right)} % the set of all mups for some concept
\def\MIPALL#1#2{\textit{MIPS}_{#1}\left(#2\right)} % the set of all mips





\begin{document}

\pagenumbering{roman}
% lets go for the title page, something like this should be okay
\begin{titlepage}
	\vspace*{2cm}
  \begin{center}
   {\huge Mining for The Perfect Movie\\}
   \vspace{2cm} 
   {\Large Student Project Data Mining HWS17\\
   Team 6\\}
   \vspace{2cm}
   {\Large Presented by \\}
   \vspace{0.5cm}
    {Steffen Jung \\
    Adrian Kochsiek \\
    Martin Koller \\
    Marvin Messenzehl \\
    Daniel Szymkowiak \\
   }
   \vspace{1cm} 
   { Submitted to the\\
    Data and Web Science Group\\
    Prof.\ Dr.\ Heiko Paulheim\\
    University of Mannheim\\} \vspace{2cm}
   {October - December 2017}
  \end{center}
\end{titlepage} 

% no lets make some add some table of contents
%\tableofcontents
\newpage

%\listofalgorithms

%\listoffigures

%\listoftables

% evntuelly you might add something like this
% \listtheorems{definition}
% \listtheorems{proposition}

%\newpage


% okay, start new numbering ... here is where it really starts
\pagenumbering{arabic}

%%%%%%%%%%%%%%%%%%%%%%%%%%%%%%%%%%%%%%%%

% INPUTS


 
\section{Problem Statement}
In today's world, each movie production comes with a lot of uncertainty for all stakeholders involved. The production of a movie usually means high upfront investment costs and the success of a movie cannot be guaranteed. That is why producing a movie is a risk for all investors.
Since a lot of factors influence the success of a movie (e.g. main actors, storyline, setting of the movie, etc.) it is a challenge to determine upfront whether it is going to be lucrative.
In order to solve this problem, Data Mining is a good technique.
Therefore, the main research of this project focuses on building a solution, which will help stakeholders to predict the success of a movie:

\begin{itemize}
    \item Will the movie be popular or will it be a flop? (based on revenue)
    \item Which factors contribute to a good movie?
\end{itemize}


\section{Data Usage}
To predict the revenue of a movie, a classifier has to be built based on a large set of movies. Therefore, the main focus will be placed on the movie dataset from Kaggle\footnote{The Movies Dataset by Rounak Banik. Can be found under \url{https://www.kaggle.com/rounakbanik/the-movies-dataset}}. This dataset contains metadata with 24 different features (e.g. budget, release date, genre and the classifier revenue) for about 45,000 movies.

In addition to the movie metadata, the cast and the crew of all contained movies will be taken into account. During the project, it will be evaluated if other data sources are necessary. Other data sources might contain IMDb movie data or official box office data with visitor numbers and specific revenues.


\section{Methodology}

The measurement of success will be based upon the created revenue of the movie. Since the revenue is a continuous attribute it will be binned into several bins/classes. The concept of binning allows to predict the expected range of revenue for a new movie.

Since the metadata of the movies already contain 24 features and not all of them are relevant, irrelevant information is going to be discarded. In order to find out the most important features, an approach of different classifiers and set of features will be applied. Movies with missing features will be filtered out beforehand. 

For the best prediction, different classifiers will be used. Since Naive Bayes gives good results, despite the assumption of independence of the features, Naive Bayes will be taken as a baseline to compare the different classifiers.

Furthermore, classifiers like k-NN, Decision Trees and Random forest will be used in this project work. For all classifiers, hyperparameter tuning will be performed, to receive the best possible results. Additionally, k-NN with k from 1 to 10 and Decision Trees with different depths and split techniques will be applied.
To compare the results of the different classifiers, it is planned to run several success measurement techniques under ten times cross-validation on the training dataset and draw ROC curves.


\section{Measurement of Success}
After finding the best classifier on the dataset with all the parameters set, a train and test split will be performed on the data. The train data won't contain any test data. Factors computed are: the accuracy, recall, precision and the F1 score of the prediction of the test data. A high F1-score will show a high success.

\section{Expected Results}
The goal of this project work is to create an application, which provides the functionality to enter all known metadata of a new movie, in order to predict the range of revenue (and therefore the success) this movie will generate. For such an application several fields of use are conceivable. This ranges from looking for hot new movies as a fan, making predictions as a market researcher or getting an overview as producer or director of a new movie.








\newpage


\pagestyle{empty}


%\section*{Ehrenw\"ortliche Erkl\"arung}
%Ich versichere, dass ich die beiliegende Master-/Bachelorarbeit ohne Hilfe Dritter
%und ohne Benutzung anderer als der angegebenen Quellen und Hilfsmittel
%angefertigt und die den benutzten Quellen w\"ortlich oder inhaltlich
%entnommenen Stellen als solche kenntlich gemacht habe. Diese Arbeit
%hat in gleicher oder \"ahnlicher Form noch keiner Pr\"ufungsbeh\"orde
%vorgelegen. Ich bin mir bewusst, dass eine falsche Er- kl\"arung rechtliche Folgen haben
%wird.
%\\
%\\

%\noindent
%Mannheim, den 31.08.2014 \hspace{4cm} Unterschrift

\end{document}
