

 
\section{Problem Statement}
 In today's world each movie production comes with a lot of uncertainty for all stakeholders involved. The production of a movie is usually very expensive and the success of a movie can not be guaranteed. Thatswhy producing a movie is a big risk for all investors. 
A lot of factors influence if a movie is going to be successful (e.g. main actors, storyline, setting of the movie...). These are too many factors, to easily tell the success of a movie. In order to solve this problem Data Mining is neccessary. 
The main research is focused on building a solution, which will help stakholders to predict the success of a movie:

\begin{itemize}
	\item Will be a movie good or will it be a flop?  (based on revenue)
	\item What influences a good movie?
\end{itemize}


\section{Data Usage}
To predict the revenue of a movie, a classifier has to be built based on a big set of movies. Therefore a big focus will be placed on the Movie Data Set from Kaggle. This dataset contains the metadata with 24 different features (e.g. budget, release-date, genre and the classifier revenue) for about 45.000 movies. Additional to the movie metadata, also the cast and the crew of all the movies are going to be taken into account. During the project it is going to be evaluated if other data sources a necessary. Other data sources might conatain: IMDB movie data/ Box office data.



\section{Methodology}

The measurement of success will be based upon the created revenue of the movie. Since the revenue is a continiuous attribute it will be binned into 10 bins/classes. The concept of binning allows to later on in which range the revenue of the new movie will be.
Since the metadata of the movies already contains 24 features and not all of them are relevant, unrelevant information is going to be discared. In order to find out the most important features, an approach of different classifier and set of features will be applied. Movies with missing features will be filtered out beforehand. \\
For the best prediction, different classifier will be used on the data set. Since Naive Bayes gives good results, despite the assumption of independence of the features, Naive Bayes will be taken as Baseline to compare the different classifier. We will work with classifier like KNN, Naive Bayes, Decision Trees and Random forest. On all the classifier some hyperparameter tuning will be performed, to get the best results. Additionally KNN with k from 1 to 10 and decision trees with with different depths and split techniques will be applied (Entropy, Gini).
To compare the results of the different classifier, it is planned to run a 10 times cross validiation on the dataset and draw a Roc-curve.


\section{Measurement of Success}
After finding the best classifier on the dataset with all the parameters set we will run a train and test split on the data. The train data won't contain any test data. We will compute the accuracy, recall, precision and the f1 score of the prediction of the test data and the prediction of the test data. A high F1-score will show a high success.

\section{Expected Results}
The goal of this project work is to create an application, which gives the the option to enter all known metadata of a new movie, in order to predict the range of revenue (and therefore the success) this movie will create.





